When adding points of elliptic curves %usually
in Montgomery form,  
one has to be careful if the points being added are equal (doubling) or not (adding) and if one of the points is the point at infinity \cite{montgomery}. 
%
Edwards curves have the advantage that there is no such case distinction and doubling can be performed with exactly the same formula as addition \cite{twisted}. 
%
In comparison, operating in Montgomery curves is cheaper. In this section, we summarize how addition and doubling is performed in both forms. 
%
For the exact number of operations required in different forms of elliptic curves, see \cite{twisted}.

\begin{itemize}
	
	\item \underline{Edwards}: 	
	% https://eprint.iacr.org/2008/013.pdf (sec 6)
	Let $\point{1}$ and $\point{2}$ be points of the Baby-Jubjub twisted Edwards elliptic curve $E$. The sum $P_1 + P_2$ is a third point $P_3 = (x_3, y_3)$ with 
		\begin{align*}
			&\lambda = d x_1x_2y_1y_2,\\
			&x_3 = (x_1y_2 + y_1x_2) / (1 + \lambda),\\
			&y_3 = (y_1y_2 - x_1x_2) / (1 - \lambda).
		\end{align*}
	Note that the neutral element is the point $O = (0,1)$ and the inverse of a point $(x,y)$ is $(-x,y)$.

	\item \underline{Montgomery}: 
	% https://link.springer.com/content/pdf/10.1007/978-3-540-46588-1_17.pdf
	Let $\point{1}\not=O$ and $\point{2}\not=O$ be two points of the Baby-JubJub elliptic curve $E_M$ in Montgomery form. 
	
	If $P_1\not=P_2$, then the sum $P_1 + P_2$ is a third point $P_3 = (x_3, y_3)$ with coordinates
	% Addition
		\begin{align}
		\label{eq-ted}
		\begin{split}
			&\Lambda = (y_2-y_1)/ (x_2-x_1),\\
			&x_3 = \Lambda^2 - A - x_1 - x_2,\\
			&y_3 = \Lambda(x_1- x_3) - y_1.
		\end{split}
		\end{align}
	%
	If $P_1 = P_2$, then $2\cdot P_1$ is a point $P_3 = (x_3, y_3)$ with coordinates
	% Doubling:
		\begin{align}
		\label{eq-mont}
		\begin{split}
			&\Lambda = (3x_1^2 + 2Ax_1 + 1)/ (2y_1),\\
			&x_3 = \Lambda^2 - A - 2x_1,\\
			&y_3 = \Lambda(x_1- x_3) - y_1.
		\end{split}	
		\end{align}
	
\end{itemize}